\chapter{Desarrollo del Algoritmo}\label{chapter:implementation}

\label{sec:16}
\subsection*{ \Large Desarrollo del Algoritmo }

 En esta sección se desarrollará el pseudocódigo del algoritmo y se analizará la convergencia y complejidad del mismo. En el siguiente gráfico se muestra el desarrollo.

 %-------------------------------
 % Main Process Algorithm
 %-------------------------------
 \begin{algorithm}[H]
 \caption{Main Process}
 \KwData{Dataset}
 \KwResult{results}
 
 \textbf{// Initialize Agents}\\
 agents.input $\leftarrow$ InitializeInputAgents()\;
 agents.output $\leftarrow$ InitializeOutputAgents()\;
 Coordinator $\leftarrow$ InitializeCoordinatorAgent()\;
 Corrector $\leftarrow$ InitializeCorrectorAgent()\;
 MaxIterations $\leftarrow$ K\;
 ConvergenceThreshold $\leftarrow$ 0.01\;
 
 \ForEach{$(X,Y)$ in Dataset}{
     InitializeInputAgentsWithX$(X)$\;
     iteration $\leftarrow$ 1\;
     converged $\leftarrow$ False\;
     stack\_edges $\leftarrow$ [ ]\;
     
     \While{iteration $\le$ MaxIterations \textbf{and} not converged}{
         \textbf{// Step 1: Coordinator determines arcs}\\
         adjacency\_matrix $\leftarrow$ Coordinator.generate\_arcs$(X)$\;
         stack\_edges.append$\bigl((\text{adjacency\_matrix},\, \text{agents.input}, \text{agents.output})\bigr)$\;
         
         \textbf{// Step 2: Check termination condition}\\
         \If{no\_arcs\_exist(adjacency\_matrix)}{
             converged $\leftarrow$ True\;
             \textbf{break}\;
         }
         
         \textbf{// Step 3: Agent computation phase}\\
         \ForEach{agent in agents}{
             agent.preconditions $\leftarrow$ calculate\_preconditions$(agent,\, \text{adjacency\_matrix})$\;
             \If{check\_activation\_conditions(agent.preconditions)}{
                 agent.output $\leftarrow$ agent.FoT(agent.preconditions, agent.inputs)\;
             }
         }
         
         delta $\leftarrow$ calculate\_max\_difference(target\_outputs, current\_outputs)\;
         \If{delta $<$ ConvergenceThreshold}{
             converged $\leftarrow$ True\;
             \textbf{break}\;
         }
         
         
         agent.inputs $\leftarrow$ agent.output\;
         
         iteration $\leftarrow$ iteration + 1\;
     }
     
         \Return AuxiliarProcess(agents, converged, adjacency\_matrix, arc\_adjustments)\;
         }
         
        \end{algorithm}

\begin{algorithm}[H]
\caption{AuxiliarProcess}
 \KwData{agents, converged, adjacency\_matrix, arc\_adjustments}
 \KwResult{results}
 
     results $\leftarrow$ CollectOutputs(agents.output)\;
     
     \If{not converged}{
        arc\_adjustments $\leftarrow$ CorrectionPhase(agents , adjacency\_matrix);
        Coordinator.update(arc\_adjustments)\;
        
        }
        \textbf{// Final output collection}
        \Return results\;

\end{algorithm}
 
 
 %-------------------------------
 % Helper Functions
 %-------------------------------

\begin{algorithm}[H]
        \caption{CorrectionPhase}
        \KwIn{agents , adjacency\_matrix}
        \KwOut{arc\_adjustments}
     \textbf{// Step 5: Correction phase}\\
         arc\_adjustments $\leftarrow$ [ ]\;
         \ForEach{iteration}{
             adjacency\_matrix, input\_agent , output\_agents  $\leftarrow$ stack\_edges[start]\;
             error $\leftarrow$ calculate\_error(output\_agents.values, Y)\;
             arc\_adjustments.append( Corrector.adjust\_arcs(error, adjacency\_matrix,agents))\;
         }
         
         
         agents.input.reset\_default\_outputs()\;
         agents.ouput.reset\_default\_inputs()\;
         
         start $\leftarrow$ 0\;
         \While{start $\le$ iteration}{
             \textbf{// Step 1: Coordinator determines arcs}\\
             adjacency\_matrix, input\_agent , output\_agents  $\leftarrow$ stack\_edges[start]\;
             
             \textbf{// Step 3: Agent computation phase}\\
             
             error $\leftarrow$ calculate\_error(output\_agents.values, Y)\;
             arc\_adjustments.append( Corrector.adjust\_arcs(error, adjacency\_matrix, agents))\;
               agents.input $\leftarrow$ agents.output
             
             start $\leftarrow$ start + 1\;
             }
         \Return arc\_adjustments\;
         
         
    \end{algorithm}
 
 \begin{algorithm}[H]
 \caption{calculate\_preconditions}
 \KwIn{agent, adjacency\_matrix}
 \KwOut{preconditions}
 preconditions $\leftarrow$ [ ]\;
 \ForEach{connected\_agent in get\_connected\_agents(agent, adjacency\_matrix)}{
     prev\_val $\leftarrow$ connected\_agent.previous\_output\;
     curr\_val $\leftarrow$ connected\_agent.current\_output\;
     max\_val $\leftarrow \max\bigl(|\text{prev\_val}|,\, |\text{curr\_val}|\bigr) + 1\times10^{-6}$\;
     precondition $\leftarrow \displaystyle \frac{|\text{prev\_val} - \text{curr\_val}|}{\text{max\_val}}$\;
     preconditions.append(precondition)\;
 }
 \Return preconditions\;
 \end{algorithm} \\
 
 \begin{algorithm}[H]
 \caption{check\_activation\_conditions}
 \KwIn{preconditions}
 \KwOut{Boolean}
 \Return all\bigl($p \geq$ ActivationThreshold for each $p$ in preconditions\bigr)\;
 \end{algorithm} \\
 
 \begin{algorithm}[H]
 \caption{CollectOutputs}
 \KwIn{OutputAgents}
 \KwOut{List of outputs}
 \Return [agent.output for each agent in OutputAgents]\;
 \end{algorithm} \\
 
 %-------------------------------
 % Corrector Agent Operations
 %-------------------------------
 \begin{algorithm}[H]
 \caption{Corrector.adjust\_arcs}
 \KwIn{error, adjacency\_matrix, agents.output, agents.input}
 \KwOut{New adjacency\_matrix}
 adjustments $\leftarrow$ []\;
 \ForEach{i,output\_agent in agents.output}{
     target $\leftarrow$ output\_agent.F$^{-1}(Y_i)$\;
     current $\leftarrow$ output\_agent.T(output\_agent.inputs)\;
     residual $\leftarrow \; |\text{target} - \text{current}|$\;
     connected\_agents $\leftarrow$ sort\_by\_relevance(agents.input)\;
     cumulative\_effect $\leftarrow$ 0\;
     updated\_weight $\leftarrow$ false\;
     cumulative\_effect $\leftarrow$ 0\;
     \ForEach{agent in connected\_agents}{
         cumulative\_effect $\leftarrow$ cumulative\_effect + agent.output\;
         adjustments.append$\bigl((\text{output\_agent},\, \text{agent})\bigr)$\;
         \If{cumulative\_effect $\geq$ residual}{
             \If{cumulative\_effect == residual}{
                 \textbf{break}\;
             }
             updated\_weight $\leftarrow$ true
             remaining $\leftarrow$ residual $-$ cumulative\_effect\;
             adjusted\_weight $\leftarrow$ remaining / last\_agent.output\;
             adjustments.append$\bigl((\text{last\_agent},\, \text{adjusted * last\_agent.output})\bigr)$\;
             update\_connection\_weight( adjustments)\;
             \textbf{break}\;
         }
     }
     \If{ not updated\_weight}{
        update\_connection\_weight(adjustments)
     }
 }
 \Return generate\_new\_adjacency\_matrix(adjustments)\;
 \end{algorithm} \\
 
 %-------------------------------
 % Coordinator Agent Operations
 %-------------------------------
 \begin{algorithm}[H]
 \caption{Coordinator.generate\_arcs}
 \KwIn{X}
 \KwOut{adjacency\_matrix}
 \Return ML\_model.predict$(X)$\;
 \end{algorithm}
 
 \begin{algorithm}[H]
 \caption{Coordinator.update}
 \KwIn{adjustments}
 training\_data.add(adjustments)\;
 retrain\_model()\;
 \end{algorithm} \\



 \section*{Análisis de convergencia}
 
    El analisis de convergencia esta pricipalemnte relacionado a los agentes corrector y coordinador, puestos que estos agentes son los que guian a los agentes 
    de entrada y salida. El agente coordinador siempre encuentra una solución correcta dada la entrada y salida de la red de las infinitas posibles, por tanto se 
    puede decir que este agente converge a la ¨solución correcta¨. En los siguiente se asume que el agente coordinador converge al mínimo global: \\

    El problema de encontrar $F:X \rightarrow Y$ se ha reducido $\underset{G}{\min}(|F(X)-Y|) \leq \epsilon$ donde $G$ es un grafo en el que sus vértices son los agentes de entrada y salida. Si 
    este el agente corrector converge a $G$  entonces la red converge a $Y$ dado $X$.


 \section*{Análisis de complejidad}

    A continuación se analisa la complejidad de cada componente del sistema y luego se sintetiza para conseguir la complejidad total. \\

    Los agentes de entrada y salida tienen complejidad $N \geq O(FoT) $ siendo $N$ el número de estos agentes el agente en cuestión, esta consideración 
    sobre los agentes de entrada y salida no se puede extender dado que dichos agentes son cualquier función que de una entrada y devuelva una salida, la 
    razón principal de la cota inferior es porque $T$ se computa en $O(N)$. El agente corrector tiene complejidad lineal con respecto al número de agentes 
    de entrada y salida, es decir $O(N)$. Sea $f$ el agente coordinador entonces este tiene complejidad $O(f)$, puesto que este es un regresor genérico. 
    El algoritmo en su totalidad tiene complejidad: 
    \begin{center}
        
        \begin{equation*}
                
                |D|(C( \max(O(F o T )) + O(f) + O(N)) + O(N)) \\
                = O(|D|C( \max(O(F o T )) + O(f) + O(N)))\\

        \end{equation*}

    \end{center}

Siendo $|D|$ el núumero de pares en el dataset y $C$ la cota máxima de iteración entre agentes.