\begin{resumen}
En un contexto de creciente interconexión global, las epidemias han evolucionado de eventos aislados a crisis que amenazan la salud pública y la economía. 
La rápida propagación de enfermedades como el COVID-19 ha revelado las debilidades en los sistemas de salud. Este estudio investiga cómo la epidemiología 
computacional, que combina matemáticas y ciencias de la computación, puede prever comportamientos epidémicos antes de que se conviertan en pandemias.Se examina
la evolución histórica de los modelos epidemiológicos, desde los enfoques del siglo XVIII hasta las técnicas modernas, destacando contribuciones significativas
como las de Daniel Bernoulli y John Snow. Los avances recientes en métodos estadísticos, especialmente la estimación bayesiana, han mejorado la capacidad para
manejar incertidumbres y ajustar parámetros en tiempo real. El trabajo presenta tres enfoques innovadores para gestionar la incertidumbre en datos: 
Arquitectura de Agentes Cooperativos, Análisis de Intervalos y Programación por Restricciones. Estas metodologías buscan optimizar la toma de decisiones en 
situaciones complejas mediante la colaboración entre agentes y el uso de representaciones intervalares. Los hallazgos indican que un marco de agentes 
cooperativos, complementado con aritmética de intervalos, puede ofrecer soluciones más efectivas que las estimaciones tradicionales. Esta investigación
 no solo avanza en el modelado epidémico, sino que también proporciona un enfoque estructurado para abordar las incertidumbres inherentes a los datos 
epidemiológicos.

\end{resumen}

\begin{abstract}
	In a context of increasing global interconnectedness, epidemics have evolved from isolated events to crises threatening public health and the economy. The rapid spread of diseases such as SARS and COVID-19 has revealed weaknesses in health systems. This study investigates how computational epidemiology, which combines mathematics and computer science, can predict epidemic behaviors before they become pandemics.
	The historical evolution of epidemiological models is examined, from 18th-century approaches to modern techniques, highlighting significant contributions from figures like Daniel Bernoulli and John Snow. Recent advances in statistical methods, particularly Bayesian estimation, have improved the ability to manage uncertainties and adjust parameters in real-time.
	The work presents three innovative approaches to managing uncertainty in data: Cooperative Agent Architecture, Interval Analysis, and Constraint Programming. These methodologies aim to optimize decision-making in complex situations through collaboration among agents and the use of interval representations. Findings indicate that a cooperative agent framework, complemented by interval arithmetic, can provide more effective solutions than traditional estimates. This research not only advances epidemic modeling but also offers a structured approach to addressing the inherent uncertainties in epidemiological data.
\end{abstract}


