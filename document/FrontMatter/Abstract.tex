\begin{resumen}
	Resumen en español
\end{resumen}

\begin{abstract}
	Resumen en inglés
\end{abstract}


In a context of increasing global interconnectedness, epidemics have evolved from isolated events to crises threatening public health and the economy. The rapid spread of diseases such as SARS and COVID-19 has revealed weaknesses in health systems. This study investigates how computational epidemiology, which combines mathematics and computer science, can predict epidemic behaviors before they become pandemics.
The historical evolution of epidemiological models is examined, from 18th-century approaches to modern techniques, highlighting significant contributions from figures like Daniel Bernoulli and John Snow. Recent advances in statistical methods, particularly Bayesian estimation, have improved the ability to manage uncertainties and adjust parameters in real-time.
The work presents three innovative approaches to managing uncertainty in data: Cooperative Agent Architecture, Interval Analysis, and Constraint Programming. These methodologies aim to optimize decision-making in complex situations through collaboration among agents and the use of interval representations. Findings indicate that a cooperative agent framework, complemented by interval arithmetic, can provide more effective solutions than traditional estimates. This research not only advances epidemic modeling but also offers a structured approach to addressing the inherent uncertainties in epidemiological data.
