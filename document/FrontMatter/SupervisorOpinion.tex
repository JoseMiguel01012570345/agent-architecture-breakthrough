\begin{opinion}

    El trabajo de diploma Un Enfoque para la Estimación de Parámetros en Condiciones
de Incertidumbre, presentado por el estudiante José Miguel Pérez Pérez para optar
por el título de licenciado en Ciencia de la Computación, se destaca como un aporte
para apoyar una necesidad crítica de un grupo de investigación de profesores de la
facultad de Matemática y Computación que trabajan la modelación, solución y análisis
de problemas aplicados a las ciencias de la vida, especialmente en biomatemática. \\

En el mismo, se retoma una investigación en la que se trabajó en dos tesis anteriores,
una de las cuales es de una de las tutoras de esta tesis, que se basan en el estudio e
implementación de un algoritmo con la aritmética intervalar, con la intención de
manejar los intervalos de definición de los parámetros como alternativa dentro del
problema de estimación de parámetros. \\

En los problemas abordados dentro de las aplicaciones matemáticas a temas de la
biociencias, es recurrente la necesidad de encontrar valores óptimos de los
parámetros que caracterizan las dinámicas de los modelos y por tanto, es un problema
bastante estudiado y abordado. Sin embargo, prevalece la necesidad de automatizar
estos procesos, especialmente cuando son variables en el tiempo o están sesgados
por la incertidumbre o la falta de información. \\

De modo que nos parecía muy útil completar las investigaciones anteriores con una
herramienta que pudiera considerar esos aspectos o al menos alguno de ellos con
estrategias computacionales actuales, para una mejor solución del Problema de
Estimación de Parámetros (PEP). \\

José Miguel llegó a nosotras hace muy poco tiempo y le dimos algunas de las líneas de
investigación de nuestro interés y se decidió por abordar el PEP combinando el uso de
aritmética de intervalos para la definición de los parámetros con las técnicas de
programación de restricciones, con el ánimo de analizar la eficiencia de esta
hibridación en modelos epidemiológicos poblacionales. De modo que podríamos
resumir que con esta tesis nos proponíamos probar el desempeño de varios enfoques
computacionales que han mostrado resultados interesantes en algunas áreas de las
matemáticas aplicadas. \\

Lo cierto es que ante nuestra idea inicial de estimar parámetros robustos
reestructurando el algoritmo intervalar para que se aplicara la programación de
restricciones y mejorar la eficiencia (respecto al tiempo) del mismo, el diplomante
propuso aplicar el concepto de ”neurona artificial” (perceptron) de redes neuronales y
nos pareció un camino interesante. \\

Pese al poco tiempo que de por sí están teniendo nuestros cursos en los últimos años y
en particular, las irregularidades que hemos enfrentado en éste de sostenidas. \\

limitaciones energéticas, consideramos que José Miguel ha cumplido con los objetivos
propuestos, logrando incluso obtener resultados con los modelos epidemiológicos
prototipos básicos SIR y SI, mostrando el potencial de esta metodología en la
estimación de cotas para los parámetros. Todo esto, gracias esencialmente a su gran
dedicación y esfuerzo, estudiando incluso temas que no están en el programa de
pregrado de su carrera. \\

Consideramos que el trabajo presentado cumple con los requerimientos para ser
defendido como tesis de licenciatura y abre nuevas vías para futuras investigaciones en
el campo de las aplicaciones biomatemáticas. \\

Le deseamos a José Miguel éxitos en su vida como profesional y muchas felicidades por
su graduación y le agradecemos por el trabajo conjunto. \\

Lic. Rocio Ortiz Gancedo \space \space \space \space \space \space \space \space \space \space Dra.Aymée de los Ángeles Marrero Severo




\end{opinion}