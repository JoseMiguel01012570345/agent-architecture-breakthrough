\printbibliography[heading=bibintoc]

\label{sec:21}

\begin{itemize}

    \begin{thebibliography}

     \bibitem[1]{Ortiz2022} {Algoritmo Intervalar: Una alternativa de solución del problema de estimación de parámetros en la Interface Eagle. Autor: Rocio Ortiz Gancedo, Tutores: Dra. Aymée Marrero Severo, Lic. Daniela González Beltrán, Trabajo de Diploma presentado en opción al título de Licenciado en Ciencia de la Computación, 2022, Universidad de la Habana Facultad Matemática y Computación.}
     \label{sec:1}

    \bibitem[2]{}\href{https://doi.org/10.1038/s41598-022-06992-0}{ articule Deep learning forecasting using time varying parameters of the SIRD model for Covid 19 tipo: JOUR, autores: Bousquet, Arthur ,Conrad, William H., Sadat, Said Omer, Vardanyan, Nelli, Hong, Youngjoon, fecha: 2022/02/22, JO: Scientific Reports, SP: 3030, volúmen 12, número de la edición: 1, número estándar   - 2045-2322, DO  - 10.1038/s41598-022-06992-0, ID: Bousquet2022.}
     \label{sec:2}

    \bibitem[3]{}  \href{https://www.redalyc.org/journal/3238/323864536006/html/}{Bayesian modeling of spatio temporal patterns of the cumulative incidence of COVID-19 in municipalities of Mexico, de Gerardo Núñez Medina, \href{https://revistarelap.org/index.php/relap}{ Revista Latinoamericana de Población}, vol. 15, núm. 28, pp. 160-178, 2021, DOI:  https://doi.org/10.31406/relap2021.v15.i1.n28.6}
      \label{sec:3}

    \bibitem[4]{}\href{https://www.semanticscholar.org/paper/Contributions-to-the-mathematical-theory-of-V.-of-a-Kermack-Mckendrick/7beba9b40b692c2daa9975861394aefddcbe602b}{ Kermack WO, McKendrick AG. Contributions to the mathematical theory of epidemics: V. Analysis of experimental epidemics of mouse-typhoid; a bacterial disease conferring incomplete immunity. Journal of Hygiene. 1939;39(3):271-288. doi:10.1017/S0022172400011918}
\label{sec:4}

    \bibitem[5]{}\href{https://personal.us.es/sergio/PDocente/lectura.pdf}{El Método de Mínimos Cuadrado, Sergio A. Cruces Álvarez, UNIVERSIDAD DE SEVILLA}
     \label{sec:5}

    \bibitem[6]{}\href{https://inria.hal.science/hal-04100467}{Daniel Bernoulli, Dominique Chapelle. Essai d'une nouvelle analyse de la mortalité causée par la petite vérole, et des avantages de l'inoculation pour la prévenir. 2023.⟨hal-04100467⟩}
\label{sec:6}

    \bibitem[7] {}\href{ https://www.google.com/books/edition/Introduction_to_Interval_Analysis/kd8FmmN7sAoC?hl=en&gbpv=0 }{Introduction to Iterval Analysis, authors:
    Ramon E. Moore, Worthington, Ohio, R. Baker Kearfott, University of Louisiana at Lafayette, Lafayette, Louisiana, Michael J. Cloud, Lawrence Technological University, Southfield, Michigan, editorial: Society for Industrial and Applied Mathematics, Philadelphia,  Publisher:Cambridge University Press, Published:April 16, 2009, ISBN:9780898716696, 0898716691,

Format:Paperback}

     \label{sec:7}

    \bibitem[8] {}\href{https://www.scielosp.org/pdf/rsap/2020.v22n2/132-137/es}{Predicciones de un modelo SEIR para casos de COVID-19 en Cali, Colombia , autores Ortega-Lenis, Delia ,Arango-Londoño, David, Muñoz, Edgar, Cuartas, Daniel E., Caicedo, Diana, Mena, Jorge, Torres, Miyerlandi, Mendez, Fabian.  Revista de Salud Pública apr 2020, Volume 22 N. 2 Pages 132 - 137,  DOI:https://doi.org/10.15446/rsap.v22n2.86432  }
     \label{sec:8}

    \bibitem [9]{}Resolución del problema de la Optimizacion Global mediante Análisis de Intervalos, Gretel Domínguez Rodríguez Tutores: Dra. Aymée Marrero Severo, MSc. Jorge Barrios Ginart, Trabajo de Diploma presentado en opción del título de Licenciado en Ciencia de la Computación, junio de 2009, Universidad de la Habana Facultad Matemática y Computación
     \label{sec:9}

    \bibitem [10]{ } \href{https://lccn.loc.gov/2019047498 }{Artificial intelligence: a modern approach / Stuart J. Russell and Peter Norvig, authors: Stuart Russell and Peter Norvig ,2020, Identifiers: LCCN 2019047498 | ISBN 9780134610993 (hardcover), Subjects: LCSH: Artificial intelligence. Classification: LCC Q335.R86 2021 | DDC 006.3–dc23, LC record available at https://lccn.loc.gov/2019047498, ScoutAutomatedPrintCode, ISBN-10: 0-13-461099-7, ISBN-13: 978-0-13-461099-3, capítulos 9,10,11}
     \label{sec:10}

    \bibitem[11]{} \href{https://ia601500.us.archive.org/19/items/b28985266/b28985266.pdf}{On the Mode of Communication of Cholera, by JOHN SNOW, M.D., JIEMBEH OF THE ROYAL COl.LEGE OF PHYSICIANS, FELLOW OF THE ROYAL MED. AND CHIR. SOCIETY, FELLOW AND VICE- PRESIDENT OF THE MEDICAL SOCIETY OF LONDON. Second Edition, LONDON JOHN CHURCHILL, NEW BURLINGTON STREET. }
     \label{sec:11}

    \bibitem[12]{}  \href{https://www.nature.com/articles/s41598-022-22945-z#citeas}{ Castro, B. M., Reis, M. M. Salles, R. M. Multi-agent simulation model updating and forecasting for the evaluation of COVID-19 transmission. Scientific Reports 12, 22091 (2022) }
     \label{sec:22}

     \bibitem[13]{}  \href{ https://doi.org/10.1093/aje/kwn118}{Title: A New Tool for Epidemiology: The Usefulness of Dynamic-Agent Models in Understanding Place Effects on Health Authors: Amy H. Auchincloss, Ana V. Diez Roux Journal: American Journal of Epidemiology. Volume: 168 ,Number: 1, Pages: 1-8 2008,  May ,DOI: 10.1093/aje/kwn118}
    \label{sec:23}

    \bibitem[15]{}  \href{https://www.cambridge.org/core/books/principles-of-constraint-programming/C008FB32571F66C3EE0EEEBDE1F98A7D}{1. Apt K. Frontmatter. In: Principles of Constraint Programming. Cambridge University Press; 2003:i-iv.}
    \label{sec:24}

    \bibitem[15]{}  \href{https://www.cambridge.org/core/books/principles-of-constraint-programming/C008FB32571F66C3EE0EEEBDE1F98A7D}{1. Apt K. Frontmatter. In: Principles of Constraint Programming. Cambridge University Press; 2003:i-iv.}
    \label{sec:25}

    \bibitem[16]{}  \href{http://saber.ucv.ve/bitstream/10872/14678/1/T026800014698-0-FinalDefensa_JoseLuisRomero-000.pdf}{UNIVERSIDAD CENTRAL DE VENEZUELA FACULTAD DE AGRONOMÍA COMISIÓN DE ESTUDIOS DE POSTGRADO POSTGRADO EN ESTADÍSTICA Métodos de Máxima Verosimilitud para la Estimación de Parámetros en Diseños
    Factoriales Mixtos pxq y sus Aplicaciones en la Agroindustria  Autor: Ing. José L. Romero Tutor: Dr. Manuel Milla Consejeros: Dr. Miguel Balza Dr. Franklin Chacín}
    \label{sec:26}

    \bibitem[17]{}\href{https://cimat.repositorioinstitucional.mx/jspui/bitstream/1008/255/2/TE%20388.pdf}{Centro de Investigaci´on en Matem´aticas, A.C. Estimación de los parámetros de un modelo haciendo uso de correspondencias con incertidumbre Tesis que para obtener el grado de Maestro en Ciencias con Especialidad en Computación y Matemáticas Industriales presenta Iván Maldonado Zambrano Director de Tesis Dr.Javier Flavio Vigueras Gómez Dr. Arturo Hernández Aguirre Guanajuato, Gto. Julio de 2011}
    \label{sec:27}

    \bibitem[18]  {}\href{https://cimat.repositorioinstitucional.mx/jspui/bitstream/1008/255/2/TE%20388.pdf}{Optimización No Lineal Basada en “Enjambre de Partículas” M. Susana Moreno, Aníbal M. Blanco, Planta Piloto de Ingeniería Química PLAPIQUI (Universidad Nacional del Sur - CONICET) Camino La Carrindanga km. 7 - 8000 Bahía Blanca - Argentina {smoreno, ablanco}@plapiqui.edu.ar}
    \label{sec:28}

    \bibitem[19]  {}\href{http://www.scielo.org.bo/scielo.php?script=sci_arttext&pid=S1562-38232020000100005&lng=es&tlng=es.}{Vargas, J. C., Cruz-Carpio, Carlos Andrés. (2020). Estudio del método Monte Carlo en simulaciones para la estimación del valor de pi. Revista Boliviana de Física, 36(36), 26-32. Recuperado en 07 de febrero de 2025}
    \label{sec:28}

    \bibitem[20]{}\href{https://www.semanticscholar.org/paper/A-Bayesian-approach-to-parameter-estimation-in-HIV-Putter-Heisterkamp/5eaa80435e642203bb1333011d1143a5b90577d2?utm_source=direct_link}{Putter, H., Heisterkamp, S.H., Lange, J.M.,  Wolf, F.D. (2002). A Bayesian approach to parameter estimation in HIV dynamical models. Statistics in Medicine, 21.}
    \label{sec:30}

    \bibitem[21]{}  Revista Ingeniería y Región Vol. 24 Julio-Diciembre 2020/Universidad Surcolombiana Artículo de Investigación Variación temporal de los índices de sensibilidad de un modelo de cultivo para jitomate en invernadero Antonio Martinez Ruíz PhD martinez.antonio@inifap.gob.mx Genaro Pérez Jiménez M.C. perez.genaro@inifap.gob.mx, Cándido Mendoza-Pérez PhD mendoza.candido@colpos.mx Felipe Roberto Flores-de la Rosa PhD. flores.felipe@inifap.gob.mx Miguel Servin Palestina servin.miguel@inifap.gob.mx Fecha de recibido:18/10/2020 Fecha de revisión:27/10/2020 Fecha de aprobación:21/12/2020 DOI: 10.25054/22161325.2833
    \label{sec:31}

    \bibitem[22]{} \href{https://theses.hal.science/INRIA/hal-04798230v1}{Physics-informed neural networks for parameter estimation and simulation of a two-group epidemiological model Kawtar Idhammou Ouyoussef, Jaafar El Karkri Leon Matar Tine and Rajae Aboulaich LERMA Laboratory, Mohammadia School of Engineering Mohammed V University in Rabat, Rabat, Morocco Inria, Universite de Lyon, 69100 Villeurbanne, France Institut Camille Jordan, CNRS UMR5208 Universite Claude Bernard Lyon 1 69603 Villeurbanne, France kawtaridhammou@research.emi.ac.ma elkarkri@emi.ac.ma leon-matar.tine@univ-lyon1.fr aboulaich@emi.ac.ma Published 19 July 2024}
    \label{sec:32}

    \bibitem[23]{} \href{https://ai.stanford.edu/~nilsson/OnlinePubs-Nils/PublishedPapers/strips.pdf}{STRIPS: A New Approach to the Application of Theorem Proving to Problem Solving' Richard E. Fikes Nils J. NHsson Stanford Research Institute, Menlo Park, California Recommended by B. Raphael Presented at the 2nd IJCAI, Imperial College, London, England, September 1-3, 1971.}
    \label{sec:33}

    \bibitem[24]{} \href{https://courses.cs.washington.edu/courses/cse473/06sp/pddl.pdf}{McDermott, D., Ghallab, M., Howe, A.E., Knoblock, C.A., Ram, A., Veloso, M.M., Weld, D.S., Wilkins, D.E. (1998). PDDL-the planning domain definition language.}
    \label{sec:34}

    \bibitem[25]{} \href{https://doi.org/10.1145/321033.321034}{Martin Davis and Hilary Putnam. 1960. A Computing Procedure for Quantification Theory. J. ACM 7, 3 (July 1960), 201–215.}
    \label{sec:35}

    \bibitem[26]{} \href{https://www.sciencedirect.com/science/article/abs/pii/0004370277900078?via%3Dihub}{Mackworth, A.K. (1977). Consistency in Networks of Relations. Artif. Intell., 8, 99-118.}
    \label{sec:36}

    \bibitem[37]{} \href{https://www.swi-prolog.org/pldoc/doc/_SWI_/library/clp/clpfd.pl}{clpfd.pl -- CLP(FD): Constraint Logic Programming over Finite Domains}
    \label{sec:27}




    \end{thebibliography}

\end{itemize}